Any BCI analysis is defined by three things: A dataset, a context, and a processing pipeline. Here we describe how all of these components are dealt with within our pipeline, and how specifically we set the options for the initial analyses presented here. 

\subsection{Datasets}

Public BCI datasets exist for a wide range of user paradigms and recording
conditions, from continuous usage to single-session to
multiple-sessions-per-subject. Within the current MOABB project, we have unified
the access to many datasets:


\begin{itemize}
\item GigaDB \cite{Cho2017}
\item Physionet\cite{Schalk2004,Goldberger2000} 
\item BNCI2020
\item BBCI \cite{Blankertz2007,Shin2017}
\item (the other BNCI2020 datasets)
\item INRIA's data
\item (?)
\end{itemize}

Adding new datasets that can be found on the internet is also simple. The MNE toolkit\cite{Gramfort2014,Gramfort2013} is used for all preprocessing and channel selection, so any dataset that can be made compatible with their framework can quickly be added to the set of data offered by this project.

\subsection{Context}

A \emph{context} is the set of characteristics that defines the preprocessing
and validation procedure. To go from a recorded EEG time-series to a pipeline
performance value for a given subject or recording session, many parameters must
be defined. Trials need to be cut out of the continuous signal and
pre-processed, and this is possible in many different ways when taking into
parameters such as trial overlap, trial length, imagery type, and more. Once the
continuous data is processed into trials, and these trials are fed into a
pipeline, the question of how to create training and test sets, and how to
report performance, comes into play. We separate these two notions in our
software and call them the \emph{paradigm} and the \emph{evaluation}
respectively.

\subsubsection{Paradigm}

A paradigm defines how one goes from continuous data to trials for a standard
machine learning algorithm to deal with. While not an issue in image processing,
it is crucial in EEG and biosignals processing because most datasets do not have
exactly the same events defined in the continuous data. For example, many motor
imagery sets use a single hand versus both feet imagery, or a single class
versus rest; likewise there are many non-motor imageries possible. For any
reasonable analysis the specific sort of imagery or ERP must be controlled for,
as they all have different characteristics in the data and further are variably
effective across subjects\cite{Scherer2015}. After choosing which events or
imageries are valid, the question comes to pre-processing of the continuous
data, in the form of ICA cleaning, bandpass filtering, and so on. These must
also be identical for valid comparisons across algorithm or datasets. Lastly, there are
questions of how to cut the data into trials: What is the trial length and
overlap; or, in the case of ERP paradigms, how long before and after the event
marker do we use? The answers to all these questions are summed up in the
paradigm object.

\subsubsection{Evaluation}

Once the data is split into trials and a pipeline is fixed, there are many ways to train and test this pipeline to minimize overfitting. For datasets with multiple subjects recorded on multiple days, we may want to determine which algorithm functions best in multi-day classification. Or, we may want to determine which algorithm is best for small amounts of data. It is easy to see that there are many possibilities for splitting data into train and test sets, and these must be fixed identically for a given analysis. Furthermore, there is the question of how to report results. Multiclass problems cannot use metrics like the ROC-AUC which provide unbiased estimates of classifier goodness in binary cases; depending on things like the class balance, various metrics have various benefits and pitfalls. Therefore this must also be fixed across all datasets, contingent on the class of predictions the pipelines attempt to make. We define this as our \emph{evaluation}.

\subsection{Pipeline}

We define a \emph{pipeline} as the processing that takes one from raw trial-wise
data into labels, taking both spatial filtering and classification model fitting
into account. A convenient API for dealing with this kind of processing is
defined by scikit-learn\cite{Pedregosa2011}, which allows for easily definable
dimensionality reduction, feature generation, and model fitting. To maximize reproducibility we allow pipelines to be defined either by yaml files or through python files that generate the objects, but force all machine learning models to follow the scikit-learn interface. 

\section{Automated Statistical Analysis}
\label{stats}
Once the scores for a set of pipelines are computed, it is important to ask what
sort of statements can be made given the results. In particular, how to decide
which pipelines perform reliably better, and what a reasonable comparison might
be, are difficult questions to answer. Given thousands of subjects, we can test
model evidence for complicated models that attempt to untangle the difference
between human internal factors (e.g. which paradigm works best for a given
person), recording factors (e.g. which preprocessing steps influence the score),
and pipeline factors. As it is, however, there is not enough data to consider
such a question for arbitrarily many pipelines. Therefore, we define the
automatic statistical pipeline as follows, to attempt to give maximally robust
results for any number of pipelines on all the current data. The goal of the
analysis is to determine which pipeline performs best, and where, as well as
which algorithms are not distinguishable from each other. In order to do this we
implemented a two-step procedure: First, the scores per dataset for a given
context are statistically tested in order to see whether there
are significant differences between algorithms. If there are, we conduct
post-hoc tests to test the ordering of algorithms. The specific steps are as
follows:

(todo: make a figure with a flowchart)

\begin{enumerate}
\item Within each dataset:
  \begin{enumerate}
  \item Run a repeated-measurements ANOVA (CITE?) over all pipelines to
    determine how likely it is that there is a difference in average performance
    between all the pipelines.
  \item If the p-value is small enough, order the pipelines by mean
    performance. For each distribution of score differences (e.g. score for best
    mean pipeline minus score for second-best, second-best minus third-best,
    etc) compute a permutation-based, paired-sample t-test to
    determine whether the distributional mean is significantly greater than
    zero.
  \item Return, for each dataset, the set of algorithms with maximum performance
    that are not statistically different from each other
  \end{enumerate}
\item Display the percentage of the time each algorithm is indistinguishable
  from the algorithm of best performance.
\end{enumerate}


\section{Experiments}

To show off the possibilities of this framework, we ran various well-known BCI
pipelines from across many papers in order to conduct the first big-data,
side-by-side analysis of the state of the art in motor imagery BCIs.

\subsection{Context}

As there are many possible motor imagery paradigms and we do not want to induce
spurious effects due to differences in imagined movement, we limit ourselves to
only right hand versus left hand imagined imagery, as they are the most
well-studied BCI paradigms\cite{Yuan2014}. For evaluations, we choose
within-session cross-validation, as this represents the best-case scenario for
any pipeline, with minimal non-stationarity. However, we note that this analysis
is equally simple for cross-subject or cross-time analysis.

\subsubsection{Paradigm}
As there are many methods that show that multiple frequency bands can lead to
improved BCI performance\cite{KaiKengAng2008}, we test two preprocessing
pipelines: A single bandpass containing both the alpha and beta ranges, from
$8-35\text{Hz}$ and another from $8-35\text{Hz}$ in 2Hz increments. The trial
length was determined by the native dataset length, and channels were varied
between all available (ranging from 3 in the case of BNCI dataset 2014-004 to 64
in the case of the Physionet data) and the largest common subset of C3 and C4.
TODO: Will we keep the C3/C4 comparison in? -- We dont need to keep it IMO

\subsubsection{Evaluation}
The evaluation was chosen to be within-session, as that minimizes the effect of non-stationarity. As this is a binary classification task, the ROC-AUC score was chosen as the metric to score 5-fold cross validation (the splits were kept identical for all pipelines). 
TODO: fix up AUC-ROC section.
\subsection{Pipelines}

Given the amount sheer breadth of models implemented within scikit-learn,
attempting an exhaustive model comparison would be impossible. Instead, we
implement a selection of pipelines from the BCI literature, as well as the
well-known standards of CSP + LDA and channel-level variances + LDA. Specific
implemented pipelines are as follows; all hyperparameters were set via cross-validation:
\begin{enumerate}
  
\item Log variance from 6 filters from regularized CSP \cite{Lotte2011}, with shrinkage LDA (CITE)
\item Tangent space projection and classification via linear SVM(CITE)
\item filter-bank CSP with mutual-information based feature selection of 10 log variances, classification via linear SVM \cite{KaiKengAng2008}
\item Log variance from 6 filters from Riemannian-mean regularized CSP + SVM
\item FM features as described in (CITE myself) in addition to channel-level variances, classification via L1-regularized logistic regression. 
\item Log variance from each channel, classified via L1-regularized logistic regression.
\end{enumerate}

%%% Local Variables:
%%% mode: latex
%%% TeX-master: "main.tex"
%%% End:
