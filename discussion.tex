We present a system for reliably comparing BCI pipelines that is both
easily extended to incorporate new datasets and equipped with an
automated statistical procedure for determining which pipelines
perform best. Furthermore, this system defines a simple interface for
submitting and validating new BCI pipelines, which could serve to
unify the many methods that exist so far. To test that system, we
present results using standard pipelines in contexts that have wide
relevance to the BCI community. By looking across multiple, large
datasets, it is possible to make statements about how BCIs perform on
average, without any sort of expert tuning of the processing chain,
and further to see where the major pitfalls still lie.

The results of this analysis suggest that many well-known methods do
not reliably out-perform simpler ones, despite the small-scale studies
done years ago to validate them. In particular, the world of CSP
regularization literature does not appear to have the effect that was
originally claimed. It shows, surprisingly, that the
major difference in BCI classification isn't actually the algorithm, as of now,
but rather the recording and human paradigm characteristics. The two
most clear findings to come out of this are that log variances on the
channel level are almost never better than CSP or Riemannian methods,
and that the tangent space classification pipeline has the best model
for single-session classification.

One crucial thing to keep in mind, looking at these results, is that
this was all done on within-session classification. Within a single
recording session the non-stationarity that has long plagued BCis is
kept to a minimum, meaning that regularization is at its least
effective. The proper conclusion would therefore not be that
regularization does not help CSP, but rather that regularization is
not necessary to combat within-recording signal non-stationarity. To
determine whether regularization helps across time, it is necessary to
do a cross-session evaluation. Unfortunately, there are fewer datasets
that have multiple sessions recorded.

The analysis here, though done with over 200 subjects, is still only a
fraction of the number of subjects recorded for BCI publications over
the years. With more papers that describe more varied setups, the
power of this system can only grow, and what this analysis shows most
clearly is that the sample size problem in BCIs is bigger than we
might have expected. By gathering the data and offering a system for
testing algorithms, we hope that this platform in the coming years can
help to solve it.

%%% Local Variables:
%%% mode: latex
%%% TeX-master: "main"
%%% End:
