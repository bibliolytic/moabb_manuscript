Brain-computer interfaces (BCIs) have long presented the neuroscience methods
community with a unique challenge. Unlike fields like vision research, where one
simply has a database of images and labels, a BCI is defined by a signal
recorded from the brain and fed into a computer, which can be influenced in any
number of ways both by the subject and by the experimenter. As a result,
validating approaches has always been a difficult task. Number of channels,
requested task, physical setup, and many other features vary between the
numerous publically available datasets online, not to mention issues of
convenience such as file format and documentation. Because of this, the BCI
methods community has long done one of two things to validate an new approach:
Recorded a new dataset, or used one of few well-known, tried-and-true datasets.

Recording a new dataset, though an elegant way to show that a proposed
method works online, presents problems for post-hoc analysis. Without
making data public, it is impossible to know whether offline
classification results are convincing or due to some coding issue or
recording artifact. Further, it is well-known that differences in
hardware \cite{Searle2000,Lopez-Gordo2014}, paradigm \cite{Allison2010}, and
subject \cite{Allison2010} can have large differences in the
outcome of a BCI task, making it very difficult to generalize findings
from any single dataset.

When recording a new dataset is not an option, One can turn to some of
the standard EEG datasets publicly available. Over the course of the
last year and a half, over a thousand journal and conference
submissions have been written on the BCI Competition III
\cite{Blankertz2006,Schloegl2005} and IV \cite{Tangermann2012}
datasets. Considering that these datasets have been available
publically for over a decade, the true number of papers which validate
results against them is simply astronomical. While it is impossible to
argue about the impact those two datasets have had on the field,
relying so heavily on a small number of datasets, with less than 50
subjects total, expose the field to several important issues. In
particular, overfitting to the setups offered there is likely.

Lastly, and possibly most problematically, the scarcity of available
code for newly published BCI algorithm puts the onus on each
individual lab to reproduce the code for all other competing methods
in order to make a claim to be comparable with the 'state-of-the-art'
(SOA). As a result, the vast majority of novel BCI algorithm papers
compare either against other work from the same lab, or old standards
such as CSP \cite{Koles1990},
with or without regularization, and LDA, or simple channel-level
variances combined with a classifier of choice \cite{Garrett2003} .

Computer vision has solved this problem with enormous datasets like
\cite{Deng2009} bundled with a reliance on a small number of software
packages to create new models, notably Tensorflow, Theano, and
Pytorch. As BCIs are inextricably linked to human use, it is not
helpful to create datasets of such size. Rather, the field requires
as many unique people recording data in many contexts in order to
create an appropriate benchmark. In contrast to image data, the goal
of BCI algorithm development is exclusively to create algorithms that
work on data that has not yet been recorded. We propose our platform, the MOABB
(Mother Of All BCI Benchmarks) Project, as a candidate for this application. 

As an initial validation of this project, we present results on the
constrained task of binary classification in two-class imagined motor
imagery, as that is the most widely used motor imagery paradigm and
allows us to demonstrate the process across the largest number of
datasets.  However, we note that this is only the first question we
attempt to answer in this field. The format allows for many other questions,
including different channel types (EEG, fNIRS, or other), multi-class
paradigms, and also transfer learning scenarios as described in
\cite{Jayaram2016}. \vj{...yeah it's a bit gratuitous. Too much?}

%%% Local Variables:
%%% mode: latex
%%% TeX-master: "main"
%%% End:
